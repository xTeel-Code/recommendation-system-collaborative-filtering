%Prerobit na jeden 200 slov obsiahly abstrakt


\documentclass{article}
\usepackage{graphicx} % Required for inserting images
\usepackage{caption} % Required for removing "Table N" label
\title{Kolaboratívne Filtrovanie na Twitteri}
\author{Samuel Benč}
\date{Október 2024}

\begin{document}

\maketitle
\begin{abstract}
Kolaboratívne filtrovanie je široko používana technika odporúčaní, \\ najmä pre sociálne siete ako je Twitter či Instagram. Využíva dáta o používateľoch ako napríklad interakcie s príspevkami, sledované účty. V twitteri je toto používane na zobrazovanie príspevkov ktoré by sa mohli páčiť na základe interakcií používateľa s predošlými príspevkami, reklamy, účty, ktoré by sa mohli páčiť používateľovi a podobne. Sledovanie interakcií užívateľov ako sa lajky, komentáre alebo prezdielanie tweetov je efektívna metóda ako pracovať s pomerne riedkymi a zreteľnými údajmi a pomerne presne.Tento dokument sa zaoberá tým, ako sa na Twitteri implementuje filtrovanie založené na spolupráci, pričom
zdôrazňuje jeho úlohu pri prispôsobovaní používateľovho zážitku vo vysoko dynamickom prostredí. Tento dokument približuje evolúciu, vysvetľuje princípy kolaboratívneho filtrovania taktiež aj rôzne typy kolaboratívneho filtrovania, približuje aj jeho problémy, porovnáva výhody a nevýhody.Spomínane sú aj kľučové výzvy pre vývojarský tím ako napríklad spracovanie masívnych dátových tokov či presnosť algoritmu.Pozrieme sa bližšie aj na možné kombinovanie dvoch typov kolaboratívneho filtrovania. Zamyslíme sa nad otázkou súkromia používateľov a etiky vývojárov. Na konci je priblížené zefektívnenie a škálovanie algoritmov na veľké dáta tzv. (Big data).
\end{abstract}
\bibliography{literatura}
\end{document}
